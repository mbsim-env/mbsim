\documentclass{report}

\usepackage{amsmath}
\usepackage[a4paper]{geometry}

\newcommand{\bs}[1]{\boldsymbol #1}

\begin{document}

We have a rotation matrix $\bs{R}$ and want the angular velocity $\bs{\omega}$. Starting from
\begin{equation}
\bs{R}=\bs{R}(\bs{q},t)
\end{equation}
we build the total derivative
\begin{equation}
\dot{\bs{R}}=\sum_i \frac{\partial \bs{R}}{\partial q_i}\dot{q}_i + \frac{\partial \bs{R}}{\partial t}
\end{equation}
Multiply from left with $\bs{R}^T$
\begin{equation}
\bs{R}^T\dot{\bs{R}}=\sum_i \bs{R}^T\frac{\partial \bs{R}}{\partial q_i}\dot{q}_i + \bs{R}^T\frac{\partial \bs{R}}{\partial t}
\end{equation}
Apply the inverse tilde operator, which transforms a skew symmetric matrix to a vector. This inverse tilde operator is distributive and linear!???
\begin{equation}
\widetilde{\bs{R}^T\dot{\bs{R}}}=\sum_i \widetilde{\left(\bs{R}^T\frac{\partial \bs{R}}{\partial q_i}\right)}\dot{q}_i + \widetilde{\bs{R}^T\frac{\partial \bs{R}}{\partial t}}
\end{equation}
Rewrite it in matrix notation
\begin{equation}
\widetilde{\bs{R}^T\dot{\bs{R}}}=\left[\widetilde{\bs{R}^T\frac{\partial \bs{R}}{\partial q_1}},\widetilde{\bs{R}^T\frac{\partial \bs{R}}{\partial q_2}},\dots\right]\dot{\bs{q}} + \widetilde{\bs{R}^T\frac{\partial \bs{R}}{\partial t}}
\end{equation}
and apply some substitutions
\begin{equation}
\underbrace{\widetilde{\bs{R}^T\dot{\bs{R}}}}_{\bs{\omega}}=\underbrace{\left[\widetilde{\bs{R}^T\frac{\partial \bs{R}}{\partial q_1}},\widetilde{\bs{R}^T\frac{\partial \bs{R}}{\partial q_2}},\dots\right]}_{\bs{J}_R}\dot{\bs{q}} + \underbrace{\widetilde{\bs{R}^T\frac{\partial \bs{R}}{\partial t}}}_{\bs{j}_R}
\end{equation}
\begin{equation}
\bs{\omega}=\bs{J}_R\cdot \dot{\bs{q}} + \bs{j}_R
\end{equation}
Hence if we define the partial derivative operator of an rotation matrix $\bs{R}$ with respect to a scalar $x$ or a vector $\bs{x}$ formally as
\begin{equation}
\text{parder}_x(\bs{R}):=\widetilde{\bs{R}^T\frac{\partial\bs{R}}{\partial x}}
\end{equation}
\begin{equation}
\text{parder}_{\bs{x}}(\bs{R}):=\left[\widetilde{\bs{R}^T\frac{\partial\bs{R}}{\partial x_1}},\widetilde{\bs{R}^T\frac{\partial\bs{R}}{\partial x_2}},\dots\right]
\end{equation}
than the rotation is fully equal to the translation and using the new fmatvec Function concept yields the following:
\begin{equation}
\bs{R}=\texttt{(*rotFunc)}(\bs{q},t)
\end{equation}
\begin{equation}
\bs{J}_R=\texttt{rotFunc->parDer1}(\bs{q},t)
\end{equation}
\begin{equation}
\bs{j}_R=\texttt{rotFunc->parDer2}(\bs{q},t)
\end{equation}
Just analog to the translations
\begin{equation}
\bs{r}=\texttt{(*transFunc)}(\bs{q},t)
\end{equation}
\begin{equation}
\bs{J}_T=\texttt{transFunc->parDer1}(\bs{q},t)
\end{equation}
\begin{equation}
\bs{j}_T=\texttt{transFunc->parDer2}(\bs{q},t)
\end{equation}

\end{document}
